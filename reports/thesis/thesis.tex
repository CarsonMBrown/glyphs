%-------------------------------------------------------------------------------
%	PACKAGES AND OTHER DOCUMENT CONFIGURATIONS
%-------------------------------------------------------------------------------

\documentclass[
11pt, % The default document font size, options: 10pt, 11pt, 12pt
oneside, % Two side (alternating margins) for binding by default, uncomment to switch to one side
english, % ngerman for German
singlespacing, % Single line spacing, alternatives: onehalfspacing or doublespacing
% draft, % Uncomment to enable draft mode (no pictures, no links, overfull hboxes indicated)
%nolistspacing, % If the document is onehalfspacing or doublespacing, uncomment this to set spacing in lists to single
liststotoc, % Uncomment to add the list of figures/tables/etc to the table of contents
toctotoc, % Uncomment to add the main table of contents to the table of contents
parskip, % Uncomment to add space between paragraphs
%nohyperref, % Uncomment to not load the hyperref package
headsepline, % Uncomment to get a line under the header
%chapterinoneline, % Uncomment to place the chapter title next to the number on one line
%consistentlayout, % Uncomment to change the layout of the declaration, abstract and acknowledgements pages to match the default layout
]{MastersDoctoralThesis} % The class file specifying the document structure

\usepackage[utf8]{inputenc} % Required for inputting international characters
\usepackage[T1]{fontenc} % Output font encoding for international characters

\usepackage{mathpazo} % Use the Palatino font by default
\usepackage{todonotes}
\usepackage{bbold}

\usepackage{listings}

\usepackage{caption}
\usepackage{subcaption}
\usepackage{float}

\usepackage[style=numeric,sorting=nyt]{biblatex}

\graphicspath{{../graphics/}}
\addbibresource{../bibliography.bib} % The filename of the bibliography

\usepackage[autostyle=true]{csquotes} % Required to generate language-dependent quotes in the bibliography

\newcommand{\chapterinput}[1]{\input{../content/chapters/#1}}
\newcommand{\appendixinput}[1]{\input{../content/appendix/#1}}
\newcommand{\bfcolor}[1]{\textbf{\color{violet}{#1}}}

\newcommand{\code}{\texttt}
\renewcommand{\O}[1]{$\mathcal{O}(#1)$}
\renewcommand{\o}[1]{$o(#1)$}
\newcommand{\T}[1]{$\Theta(#1)$}
\newcommand{\W}[1]{$\Omega(#1)$}
\newcommand{\w}[1]{$\omega(#1)$}
\newcommand{\command}[1]{\\\\\code{#1}\\\\\noindent}

\newcommand{\seefig}[1]{(Figure \ref{fig:#1})}
\newcommand{\seeeq}[1]{(Equation \ref{eq:#1})}
\newcommand{\codefigsize}{footnotesize}

%-------------------------------------------------------------------------------
%	MARGIN SETTINGS
%-------------------------------------------------------------------------------

\geometry{
	paper=a4paper, % Change to letterpaper for US letter
	inner=2.5cm, % Inner margin
	outer=3.8cm, % Outer margin
	bindingoffset=.5cm, % Binding offset
	top=1.5cm, % Top margin
	bottom=1.5cm, % Bottom margin
	%showframe, % Uncomment to show how the type block is set on the page
}

%-------------------------------------------------------------------------------
%	THESIS INFORMATION
%-------------------------------------------------------------------------------


\thesistitle{Automated Optical Detection and Recognition of Greek Letters on Degraded Papyri} % Your thesis title, this is used in the title and abstract, print it elsewhere with \ttitle
\supervisor{Dr. Mark-Jan \textsc{Nederhof}} % Your supervisor's name, this is used in the title page, print it elsewhere with \supname
\degree{Master in Science - Computer Science} % Your degree name, this is used in the title page and abstract, print it elsewhere with \degreename
\author{Carson \textsc{Brown}} % Your name, this is used in the title page and abstract, print it elsewhere with \authorname

\subject{Computer Science} % Your subject area, this is not currently used anywhere in the template, print it elsewhere with \subjectname
\university{\href{https://www.st-andrews.ac.uk/}{University of St Andrews}} % Your university's name and URL, this is used in the title page and abstract, print it elsewhere with \univname
\department{\href{https://www.st-andrews.ac.uk/computer-science/}{Department of Computer Science}} % Your department's name and URL, this is used in the title page and abstract, print it elsewhere with \deptname

\AtBeginDocument{
\hypersetup{pdftitle=\ttitle} % Set the PDF's title to your title
\hypersetup{pdfauthor=\authorname} % Set the PDF's author to your name
}

\begin{document}

\frontmatter % Use roman page numbering style (i, ii, iii, iv...) for the pre-content pages

\pagestyle{plain} % Default to the plain heading style until the thesis style is called for the body content

%-------------------------------------------------------------------------------
%	TITLE PAGE
%-------------------------------------------------------------------------------

\begin{titlepage}
\begin{center}

\vspace*{.06\textheight}
{\scshape\LARGE \univname\par}\vspace{1.5cm} % University name
\textsc{\Large Masters Thesis}\\[0.5cm] % Thesis type

\HRule \\[0.4cm] % Horizontal line
{\huge \bfseries \ttitle\par}\vspace{0.4cm} % Thesis title
\HRule \\[1.5cm] % Horizontal line

\begin{minipage}[t]{0.4\textwidth}
\begin{flushleft} \large
\emph{Author:}\\\authorname
\end{flushleft}
\end{minipage}
\begin{minipage}[t]{0.4\textwidth}
\begin{flushright} \large
\emph{Supervisor:}\\\supname
\end{flushright}
\end{minipage}\\[1cm]

\includegraphics[width=0.5\textwidth]{{misc/st-andrews_vertical_black.png}}\\[1cm]

% \large \textit{A thesis submitted in fulfillment of the requirements\\ for the degree of \degreename}\\[0.3cm] % University requirement text
% \textit{in the}\\[0.4cm]
% \deptname\\[2cm] % Research group name and department name

{\large \today} % Date

\vfill

\end{center}
\end{titlepage}

%-------------------------------------------------------------------------------
%	ABSTRACT PAGE
%-------------------------------------------------------------------------------

\addchaptertocentry{Abstract} % Add the abstract to the table of contents

\begin{abstract}
	Annotating and transcribing papyri is a time-consuming and complicated process that requires expertise in both language and papyrus to correctly transcribe damaged texts. This report proposes a fully autonomous pipeline that takes digital images of papyrus and outputs bounding boxes and classifications for each glyph, as well as grouping glyphs into potential lines. The pipeline consists of a binarization step, where glyphs are digitally separated from the papyrus using various methods, a glyph bounding step, a line grouping step, and finally a classification step, each with options that can be used to increase recall or precision. The processes for generating the pipeline and various alternative approaches are discussed, along with an evaluation of the pipeline and its component modules.
\end{abstract}

%-------------------------------------------------------------------------------
%	ACKNOWLEDGEMENTS
%-------------------------------------------------------------------------------

\addchaptertocentry{Acknowledgements} % Add the acknowledgements to the table of contents
\begin{acknowledgements}
	I would like to thank my project supervisor, Dr. Mark-Jan Nederhof, for his insights on language and character recognition and his help on this project. I would also like to thank Dr. Isabelle Marthot-Santaniello for her assistance with understanding the dataset. Finally, I would like to thank my family and friends who have supported me throughout my education; your support is invaluable.
\end{acknowledgements}

%-------------------------------------------------------------------------------
%	DECLARATION PAGE
%-------------------------------------------------------------------------------

\addchaptertocentry{Declaration} % Add the declaration to the table of contents
\begin{declaration}
I declare that the material submitted for assessment is my own work except where credit is explicitly given to others by citation or acknowledgement. This work was performed during the current academic year except where otherwise stated. The main text of this project report is 12,782 words long, including project specification and plan. In submitting this project report to the University of St Andrews, I give permission for it to be made available for use in accordance with the regulations of the University Library. I also give permission for the title and abstract to be published and for copies of the report to be made and supplied at cost to any bona fide library or research worker, and to be made available on the World Wide Web. I retain the copyright in this work.

\noindent Signed: \textit{Carson Brown}

\noindent Date: \textit{\today}
\end{declaration}

%-------------------------------------------------------------------------------
%	QUOTATION PAGE
%-------------------------------------------------------------------------------

% \vspace*{0.2\textheight}
%
% \noindent\enquote{\itshape “The only true wisdom is in knowing you know nothing.}\bigbreak
%
% \hfill Socrates

% \noindent\enquote{\itshape The direction in which education starts a man will determine his future life.}\bigbreak
%
% \hfill Plato

% TODO: THIS ONE
% \noindent\enquote{\itshape You should not honor men more than truth.}\bigbreak
%
% \hfill Plato

%-------------------------------------------------------------------------------
%	LIST OF CONTENTS/FIGURES/TABLES PAGES
%-------------------------------------------------------------------------------

\tableofcontents % Prints the main table of contents

\listoffigures % Prints the list of figures

% \listoftables % Prints the list of tables

%-------------------------------------------------------------------------------
%	ABBREVIATIONS
%-------------------------------------------------------------------------------

\appendixinput{abbreviations.tex}

%-------------------------------------------------------------------------------
%	SYMBOLS
%-------------------------------------------------------------------------------

\appendixinput{symbols.tex}

%-------------------------------------------------------------------------------%	PHYSICAL CONSTANTS/OTHER DEFINITIONS
%-------------------------------------------------------------------------------

\appendixinput{constants.tex}

%-------------------------------------------------------------------------------
%	DEDICATION
%-------------------------------------------------------------------------------

% \dedicatory{For/Dedicated to/To my\ldots}

%-------------------------------------------------------------------------------
%	MAIN CONTENT
%-------------------------------------------------------------------------------

\mainmatter % Begin numeric (1,2,3...) page numbering

\pagestyle{thesis}
% Total Words: (Jan 5)
% 12,782

% This doc: 307 words (Jan 5)
% Includes abstract, declaration, dedication

% 316 Words (Jan 5)
\chapter{Introduction}
\chapterinput{introduction.tex}

% 1895 Words (Jan 5)
\chapter{Context Survey}
\chapterinput{context_survey.tex}

% 31 Words (Jan 5)
\chapter{Ethical Considerations}
\chapterinput{ethics.tex}

% 151 Words (Jan 5)
\chapter{Requirement Specification}
\chapterinput{requirements.tex}

% 428 Words (Jan 5)
\chapter{Software Engineering Methodology}
\chapterinput{software_engineering.tex}

% 551 Words (Jan 5)
\chapter{Design}
\chapterinput{design.tex}

% Up to this point
% 3,679 (Jan 5)

% 4,187 Words (Jan 5)
% + 1,472 in figures (Jan 5)
\chapter{Implementation}
\chapterinput{implementation.tex}

% 2,470 Words (Jan 4)
% + 454 in figures (Jan 4)
\chapter{Evaluation}
\chapterinput{evaluation.tex}

% After this point
% 315 Words (Jan 5)

% 110 Words (Jan 5)
\chapter{Conclusion}
\chapterinput{conclusion.tex}

% 205 Words (Dec 15)
\chapter{Proposed Future Work}
\chapterinput{future.tex}

%-------------------------------------------------------------------------------
%	APPENDIX
%-------------------------------------------------------------------------------

\appendix
% 205 Words (Jan 5)

% 105 Words (Jan 4)
\chapter{File Fragments}
\appendixinput{code.tex}

% 50 Words (Dec 15)
\chapter{Neural Network Architecture}
\appendixinput{nnArchitectures.tex}

% 50 Words (Dec 15)
\chapter{Convolutional Neural Network Architecture}
\appendixinput{cnnArchitectures.tex}

%-------------------------------------------------------------------------------
%	BIBLIOGRAPHY
%-------------------------------------------------------------------------------

\emergencystretch=1em

% \nocite{*}

\printbibliography[heading=bibintoc]

%-------------------------------------------------------------------------------

\end{document}
