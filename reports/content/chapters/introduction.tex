Papyrus was used in the ancient world as early as the fourth millennium BCE as a writing surface, behaving similarly to a thick paper \cite{Houston}. Constructed from the pith of the papyrus plant, multiple sheets were often joined with each other to form scrolls. Although originally from Egypt, ancient papyri originated from areas throughout the Mediterranean region containing writings detailing topics from the Gospel of John to remedial methods for scorpion poison to descriptions of demonic and mystical spells \cite{Comfort, Scorpion, Betz}. In addition to containing such a vast array of information from the ancient world, these texts are often written in Greek, Coptic, and Hebrew in addition to the Egyptian Hieroglyphic, Hieratic, and Demotic.

With such a wide array of topics and languages, it takes significant amounts of time, resources, and expertise to accurately transcribe and catalog the information contained on even a single sheet of papyrus. With hundreds of thousands, if not millions of surviving papyrus artifacts contained within both museums and private collections, a pipeline that could utilize existing resources to assist papyrologists with this task would allow for "major steps forward" in the field \cite{Contest}. With papyrus being a "very challenging type of historical document" to digitally transcribe, there are not yet any reliable tools that address this problem, which this project, inspired by the contest \cite{Contest} organized by Dr. Isabelle Marthot-Santaniello, seeks to remedy.

This project seeks to automate two separate-but-connected tasks, the localization of glyphs and the classification of glyphs. Using a dataset of images of papyrus from multiple collections which is representative of the diversity of Greek papyri artifacts this project hopes to create a general solution to the problem of automating the transcribing of ancient Greek papyri samples \cite{Contest}. This project makes use of optical character recognition and natural language processing techniques in addition to exploring the use of convolutional neural networks, among other techniques, to classify and localize glyphs.
