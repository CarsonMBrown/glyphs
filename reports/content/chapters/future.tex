\todo{NEW}

Further research on automatically annotating glyphs on papyrus could focus on three different areas. One potential direction for future study could be to develop more sophisticated methods for binarization by using a RCNN type approach to locate and bound the glyphs before utilizing binarizing techniques on them. This could decrease the amount of noise that would need to be managed by the binarization and could allow for a higher signal to noise ratio in the resultant binarizations. Composite images could then be reconstructed from these glyph binarizations, which would allow later parts of the pipeline to process complete images as normal.

Another area for further research could be to expand the scope of the study to include more diverse examples of glyphs and papyrus texts. This could involve examining a wider range of texts from different regions or time periods or studying texts in different languages or writing systems. This could allow for a more robust glyph bounding system and may allow for more features of the glyphs to be classified, such as diacritics, case, and preservation level.

Finally, existing texts could be utilized further by allowing pipeline to find similar words or full lines of text to the ones being annotated to correct glyph classification errors.
