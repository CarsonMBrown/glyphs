The following figures are psuedocode simplifications of the implementations of the convolutional neural networks utilized in the project. The full code for these networks is supplied in the github repo \cite{Git} of the project, but these should act as quick references for the differences in depth and layers between the networks.

\begin{figure}
  \caption{AlexNet LSTM Classifier}
  \label{fig:cnnAlexNetLSTM}
  \par\noindent\rule{\textwidth}{0.5pt}
  \begin{\codefigsize}
  \begin{lstlisting}
Conv2d(3, 64, kernel_size=11, stride=4, padding=2),
ReLU(),
MaxPool2d(kernel_size=3, stride=2),
Conv2d(64, 192, kernel_size=5, padding=2),
ReLU(),
MaxPool2d(kernel_size=3, stride=2),
Conv2d(192, 384, kernel_size=3, padding=1),
ReLU(),
Conv2d(384, 256, kernel_size=3, padding=1),
ReLU(),
Conv2d(256, 256, kernel_size=3, padding=1),
ReLU(),
MaxPool2d(kernel_size=3, stride=2),
self.avgpool = AdaptiveAvgPool2d((6, 6))
Dropout( bidirectional.1),
Linear(256 * 6 * 6, 4096),
ReLU(),
Dropout( bidirectional.1),
Linear(4096, 4096),
ReLU(),
Linear(4096, 2048),
LSTM(2048, 24, bidirectional=True)
Linear(48, 24)
  \end{lstlisting}
  \end{\codefigsize}
  \par\noindent\rule{\textwidth}{0.5pt}
\end{figure}

\begin{figure}
  \caption{ResNeXt Simple-LSTM Classifier}
  \label{fig:cnnResNeXt50SimpleLSTM}
  \par\noindent\rule{\textwidth}{0.5pt}
  \begin{\codefigsize}
  \begin{lstlisting}
self.resnext = resnext50_32x4d(ResNeXt50_32X4D_Weights.IMAGENET1K_V2)
self.resnext.fc = LSTM(self.resnext.fc.in_features, 24, bidirectional=True)
self.classifier = Linear(48, 24)
  \end{lstlisting}
  \end{\codefigsize}
  \par\noindent\rule{\textwidth}{0.5pt}
\end{figure}

\begin{figure}
  \caption{ResNeXt50 Linear-to-LSTM Classifier}
  \label{fig:cnnResNeXt50Linear2LSTM}
  \par\noindent\rule{\textwidth}{0.5pt}
  \begin{\codefigsize}
  \begin{lstlisting}
self.resnext = resnext50_32x4d(ResNeXt50_32X4D_Weights.IMAGENET1K_V2)
self.resnext.fc = Sequential(
    Linear(self.resnext.fc.in_features, 200),
    Dropout(0.1),
    ReLU(),
    Linear(200, 100),
    Dropout(0.1),
    ReLU(),
    LSTM(100, 48, bidirectional=True)
),
Linear(48, 24)
  \end{lstlisting}
  \end{\codefigsize}
  \par\noindent\rule{\textwidth}{0.5pt}
\end{figure}

\begin{figure}
  \caption{ResNeXt50 LSTM-to-Linear Classifier}
  \label{fig:cnnResNeXt50LSTM2Linear}
  \par\noindent\rule{\textwidth}{0.5pt}
  \begin{\codefigsize}
  \begin{lstlisting}
self.resnext = resnext50_32x4d(ResNeXt50_32X4D_Weights.IMAGENET1K_V2)
self.resnext.fc = LSTM(self.resnext.fc.in_features, 48, bidirectional=True)
Dropout(.2),
ReLU(),
Linear(96, 96),
Dropout(.2),
ReLU(),
Linear(96, 24)
  \end{lstlisting}
  \end{\codefigsize}
  \par\noindent\rule{\textwidth}{0.5pt}
\end{figure}

\begin{figure}
  \caption{ResNeXt50 LSTM-to-Tall-Linear Classifier}
  \label{fig:cnnResNeXt50LSTM2TallLinear}
  \par\noindent\rule{\textwidth}{0.5pt}
  \begin{\codefigsize}
  \begin{lstlisting}
self.resnext = resnext50_32x4d(ResNeXt50_32X4D_Weights.IMAGENET1K_V2)
self.resnext.fc = LSTM(self.resnext.fc.in_features, 96, bidirectional=True)
Dropout(0.2),
ReLU(),
Linear(192, 96),
Dropout(0.2),
ReLU(),
Linear(96, 24)
  \end{lstlisting}
  \end{\codefigsize}
  \par\noindent\rule{\textwidth}{0.5pt}
\end{figure}

\begin{figure}
  \caption{MNIST9975 CNN Classifier}
  \label{fig:cnnMNISTCNN}
  \par\noindent\rule{\textwidth}{0.5pt}
  \begin{\codefigsize}
  \begin{lstlisting}
Conv2d(3, 32, 3, padding=1),
ReLU(),
BatchNorm2d(32),
Conv2d(32, 32, 3, padding=1),
ReLU(),
BatchNorm2d(32),
Conv2d(32, 32, 3, stride=2, padding=1),
ReLU(),
BatchNorm2d(32),
MaxPool2d(2, 2),
Dropout(.5),
Conv2d(32, 64, 3, padding=1),
ReLU(),
BatchNorm2d(64),
Conv2d(64, 64, 3, padding=1),
ReLU(),
BatchNorm2d(64),
Conv2d(64, 64, 3, stride=2, padding=1),
ReLU(),
BatchNorm2d(64),
MaxPool2d(2, 2),
Dropout(.5),
Conv2d(64, 128, 3, padding=1),
ReLU(),
BatchNorm2d(128),
MaxPool2d(2, 2),
Dropout(.5),
Linear(128, 24)
  \end{lstlisting}
  \end{\codefigsize}
  \par\noindent\rule{\textwidth}{0.5pt}
\end{figure}

\begin{figure}
  \caption{MNIST9975 LSTM Classifier}
  \label{fig:cnnMNISTCNNLSTM}
  \par\noindent\rule{\textwidth}{0.5pt}
  \begin{\codefigsize}
  \begin{lstlisting}
Conv2d(3, 32, 3, padding=1),
ReLU(),
BatchNorm2d(32),
Conv2d(32, 32, 3, padding=1),
ReLU(),
BatchNorm2d(32),
Conv2d(32, 32, 3, stride=2, padding=1),
ReLU(),
BatchNorm2d(32),
MaxPool2d(2, 2),
Dropout(.5),
Conv2d(32, 64, 3, padding=1),
ReLU(),
BatchNorm2d(64),
Conv2d(64, 64, 3, padding=1),
ReLU(),
BatchNorm2d(64),
Conv2d(64, 64, 3, stride=2, padding=1),
ReLU(),
BatchNorm2d(64),
MaxPool2d(2, 2),
Dropout(.5),
Conv2d(64, 128, 3, padding=1),
ReLU(),
BatchNorm2d(128),
MaxPool2d(2, 2),
Dropout(.5),
LSTM(128, 24, bidirectional=True),
nn.Linear(48, 24)
  \end{lstlisting}
  \end{\codefigsize}
  \par\noindent\rule{\textwidth}{0.5pt}
\end{figure}

\begin{figure}
  \caption{MNIST9975 LSTM-to-Linear Classifier}
  \label{fig:cnnMNISTCNNLSTM2Linear}
  \par\noindent\rule{\textwidth}{0.5pt}
  \begin{\codefigsize}
  \begin{lstlisting}
Conv2d(3, 32, 3, padding=1),
ReLU(),
BatchNorm2d(32),
Conv2d(32, 32, 3, padding=1),
ReLU(),
BatchNorm2d(32),
Conv2d(32, 32, 3, stride=2, padding=1),
ReLU(),
BatchNorm2d(32),
MaxPool2d(2, 2),
Dropout(.5),
Conv2d(32, 64, 3, padding=1),
ReLU(),
BatchNorm2d(64),
Conv2d(64, 64, 3, padding=1),
ReLU(),
BatchNorm2d(64),
Conv2d(64, 64, 3, stride=2, padding=1),
ReLU(),
BatchNorm2d(64),
MaxPool2d(2, 2),
Dropout(.5),
Conv2d(64, 128, 3, padding=1),
ReLU(),
BatchNorm2d(128),
MaxPool2d(2, 2),
Dropout(.5),
LSTM(128, 24, bidirectional=True),
nn.Linear(48, 48),
nn.Linear(48, 48),
nn.Linear(48, 24)
  \end{lstlisting}
  \end{\codefigsize}
  \par\noindent\rule{\textwidth}{0.5pt}
\end{figure}
