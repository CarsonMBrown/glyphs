\documentclass[12pt,a4paper,final]{article}
\usepackage[a4paper, total={6in, 8.5in}]{geometry}
\setcounter{secnumdepth}{0}
\usepackage[utf8]{inputenc}
\usepackage{amsmath}
\usepackage{amsfonts}
\usepackage{amssymb}
\usepackage{sectsty}
\allsectionsfont{\centering}
\usepackage{textcomp}
\usepackage{pdfpages}
\usepackage{graphicx}
\usepackage{subcaption}
\graphicspath{{../figures/}}
\usepackage{hyperref}
\usepackage{enumitem}
\usepackage{multicol}
\usepackage{forest}
\usepackage{float}
\usepackage{mathtools}
\usepackage{todonotes}
\usepackage{fancyhdr}
\usepackage{tikz}
\usepackage[style=numeric,sorting=nyt]{biblatex}
\makeatletter
\Hy@AtBeginDocument{
  \def\@pdfborder{0 0 1}
  \def\@pdfborderstyle{/S/U/W 1}
}
\makeatother
\hypersetup{
    colorlinks=true,
    citecolor=black,
    citebordercolor=black,
    linkcolor=blue,
    filecolor=magenta,
    urlcolor=cyan
}

\urlstyle{same}

\newcommand{\code}{\texttt}
\renewcommand{\O}[1]{$\mathcal{O}(#1)$}
\renewcommand{\o}[1]{$o(#1)$}
\newcommand{\T}[1]{$\Theta(#1)$}
\newcommand{\W}[1]{$\Omega(#1)$}
\newcommand{\w}[1]{$\omega(#1)$}
\newcommand{\command}[1]{\\\\\code{#1}\\\\\noindent}

\newcommand\qed{$\blacksquare$}

\DeclarePairedDelimiter\ceil{\lceil}{\rceil}
\DeclarePairedDelimiter\floor{\lfloor}{\rfloor}
\DeclarePairedDelimiter\abs{\lvert}{\rvert}

\addbibresource{{../bibliography.bib}}

\newcommand{\practicalInfoVar}{CS5199 - Context Survey}
\newcommand{\titleVar}{Thesis Skeleton, Context Survey, \& Work Plan}
\newcommand{\subTitleVar}{\\}
\newcommand{\dateVar}{October 14, 2022}
\newcommand{\matricNumberVar}{180008859}

\begin{document}
\title{\practicalInfoVar:\\\titleVar\subTitleVar}
\author{\matricNumberVar}
\date{\dateVar}
\maketitle

\pagestyle{fancy}
\fancyhf{}
\lhead{\practicalInfoVar}
\rhead{| \matricNumberVar}
\rfoot{Page \thepage}

\newpage
\tableofcontents
\newpage

\setlength{\parskip}{1em}

\section{Proposed Thesis Skeleton}
\todo{}
% table of contents with all the chapter and section headings. These will form the skeleton of your thesis and ensure that your report is properly structured;
\begin{itemize}[label={}]
  \item Abstract
  \item Acknowledgements
  \item Declaration
  \item Table of Contents
  \item Introduction
  \item Context Survey
  \item - Binarization
  \item - Noise
  \item - Glyph Grouping
  \item - Classification
  \item - Existing Methodologies
  \item Objectives
  \item - Primary
  \item - Secondary
  \item Design / Implementation
  \item - [Section for each method attempted]
  \item Evaluation
  \item - [Section for each method attempted]
  \item Conclusion
  \item References
  \item Appendix
\end{itemize}

\newpage
\section{Context Survey}
% a mostly complete review of related work (literature review). This is normally 5-10 pages long and will include citations to most important papers on this topic and explain how they relate to the task;
\subsection{Binarization / Filtering}
\todo{}

\subsection{Noise Reduction}
\todo{}

\subsection{Glyph Grouping / Line Formation}
\todo{}

\subsection{Classification}
\todo{}

\subsection{Existing Methods}
\todo{}

\newpage
\section{Work Plan}
\todo{}
% a work plan for the rest of your dissertation period (week-by-week) indicating the main tasks and objectives you will need to tackle and when you will be doing this. This is usually in the form of a table, a Gantt chart, or similar.

\printbibliography[heading=bibintoc, title={References}]

\end{document}
