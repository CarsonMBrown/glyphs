\documentclass[12pt,a4paper,final]{article}
\usepackage[a4paper, total={6in, 8.5in}]{geometry}
\setcounter{secnumdepth}{0}
\usepackage[utf8]{inputenc}
\usepackage{amsmath}
\usepackage{amsfonts}
\usepackage{amssymb}
\usepackage{sectsty}
\allsectionsfont{\centering}
\usepackage{textcomp}
\usepackage{pdfpages}
\usepackage{graphicx}
\usepackage{subcaption}
\graphicspath{{../figures/}}
\usepackage{hyperref}
\usepackage{enumitem}
\usepackage{multicol}
\usepackage{forest}
\usepackage{float}
\usepackage{mathtools}
\usepackage{todonotes}
\usepackage{fancyhdr}
\usepackage{tikz}
\usepackage[style=numeric,sorting=nyt]{biblatex}
\makeatletter
\Hy@AtBeginDocument{
  \def\@pdfborder{0 0 1}
  \def\@pdfborderstyle{/S/U/W 1}
}
\makeatother
\hypersetup{
    colorlinks=true,
    citecolor=black,
    citebordercolor=black,
    linkcolor=blue,
    filecolor=magenta,
    urlcolor=cyan
}

\urlstyle{same}

\addbibresource{../bibliography.bib}

\newcommand{\code}{\texttt}
\renewcommand{\O}[1]{$\mathcal{O}(#1)$}
\renewcommand{\o}[1]{$o(#1)$}
\newcommand{\T}[1]{$\Theta(#1)$}
\newcommand{\W}[1]{$\Omega(#1)$}
\newcommand{\w}[1]{$\omega(#1)$}
\newcommand{\command}[1]{\\\\\code{#1}\\\\\noindent}

\newcommand\qed{$\blacksquare$}

\DeclarePairedDelimiter\ceil{\lceil}{\rceil}
\DeclarePairedDelimiter\floor{\lfloor}{\rfloor}
\DeclarePairedDelimiter\abs{\lvert}{\rvert}

\newcommand{\practicalInfoVar}{CS5199 - Context Survey}
\newcommand{\titleVar}{X}
\newcommand{\subTitleVar}{\\}
\newcommand{\dateVar}{September 23, 2022}
\newcommand{\matricNumberVar}{180008859}

\begin{document}
\title{\practicalInfoVar:\\\titleVar\subTitleVar}
\author{\matricNumberVar}
\date{\dateVar}
\maketitle

\pagestyle{fancy}
\fancyhf{}
\lhead{\practicalInfoVar}
\rhead{| Matriculation Number: \matricNumberVar}
\rfoot{Page \thepage}

\newpage
\tableofcontents
\newpage

\setlength{\parskip}{1em}

% This is document of approximately 2 pages which should be discussed with the supervisor before uploading to MMS. The aim is to identify the problem (description), what needs to be achieved to solve it (objectives), discussion of any ethical concerns (ethics) and what resources are required (resources). There is a template on studres. This document forms the baseline for starting work.

\section{Description}
% Description should be two or three paragraphs that clearly describe the problem and the proposed work.

% In the Description, you cite [1] a lot. That is necessary if you largely quote that text. But if these are literal quotes, these sentences should be in quotation marks.
% Generally, it is better to first gain the understanding from a text such as [1], then to put that text aside, and write up a report (or DOER or whatever you need to write), from your own understanding and in your own words.

This project aims to solve the task of a recent competition: to detect occurrences of Greek letters and to classify them. Specifically, to investigate both glyph detection and glyph classification on Greek papyri, a "very challenging type of historical document"\cite{contest}.

This project seeks to automate two different tasks, the localization of glyphs and the classification of glyphs. "With document images from several institutions, representative of the diversity of book hands on papyri (a millennium time span, various script styles, provenance, states of preservation, means of digitization and resolution)"\cite{contest} this project hopes to create a general solution to the problem of automating the reading of ancient Greek papyri samples. This project will make use of optical character recognition and natural language processing techniques as well as investigate the use of convolutional neural networks, image distortion, and image filtering to classify and localize glyphs.

Normally requiring the expertise of trained papyrologists, this program could vastly increase both the efficiency with which papyri can be read and cataloged, but also allow for this task to be done by anyone, without requiring an expert to be available. This would allow more researches access to the information contained on the papyri, increasing the amount of information that can be extracted from this mostly untapped resource. Additionally, this project could bring ancient Greek papyri closer to being as documented and researched as papyri containing ancient Hebrew, and ancient Egyptian.

\section{Objectives}
% Objectives should list specific, measurable targets that can be used to structure your work. Once agreed, you cannot change the objectives without agreement of the supervisor. The objectives should be listed in order of importance and divided into 3-4 primary and 3-4 secondary objectives. Primary objectives should be realistic and required for a successful project. Secondary objectives will usually be more exciting and open-ended things to tackle once the primary objectives have been completed. Occasionally there are also tertiary objectives, but these are rare.

\subsection{Primary Objectives}
Investigate the following...

\begin{itemize}
  \item taking in an image of a single sheet of ancient papyrus with Greek letters written on it return to the user the Greek letters that appear on the papyri. (in json or similar format)
  \item taking in an image of a single sheet of ancient papyrus with Greek letters written on it and return to the user the bounding boxes of the Greek letters that appear on the papyri. (in json or similar format)
  \item separating the ink in the image from the rest of the image using several methods (CNNs, thresholding, noise removal, pixel clustering) (binarizing) and save the resultant image.
\end{itemize}

\subsection{Secondary Objectives}

\begin{itemize}
  \item investigate bounding the diacritic signs of the characters and see if these can be automatically classified as well.
\end{itemize}

Investigate and compare some or all of the following to generate a pipeline.

\begin{itemize}
  \item CNN based methods for glyph bounding, separation, and/or classification
  \item image distortion methods for glyph classification
  \item stochastic model based methods for glyph classification
  \item algorithmic based methods for glyph bounding and/or separation
\end{itemize}

\section{Ethics}
% The ethics section of the DOER document should list any possible ethical concerns. This must be discussed with your supervisor who will have more experience.
There are no ethical concerns for this project.

\section{Resources}
% The resources section should list any special resources that go beyond normal school provision. The school labs will be at your disposal and this will cover the needs for most projects, but occasionally further resources are needed: special software or hardware, working space, etc. Please list everything that is needed here (be realistic!) and who is expected to source it. Also, after discussing this with your supervisor contact fixit so they can consider your request and start working on it.
No special resources beyond those already provided by the school should be required.

\printbibliography[heading=bibintoc, title={References}]

\end{document}
