\begin{figure}
  \caption{Minimal Example of the CSV Output Format}
  \label{fig:csv_output}
  \par\noindent\rule{\textwidth}{0.5pt}
  \begin{\codefigsize}
  \lstset{literate={∆}{{$\Delta$}}1}
  \begin{lstlisting}
    glyph,  certainty, min_x,  min_y,  max_x,  max_y,        image_path
        ∆,      0.253,   100,     50,     20,     30,  images/image.png
  \end{lstlisting}
  \end{\codefigsize}
  \par\noindent\rule{\textwidth}{0.5pt}
  The minimum required information for a single image with a single instance of the $\Delta$ glyph in the CSV output file format. The bounding boxes in this format are output in the Pascal VOC bounding box format \seefig{pascalbbox}, in which bounding boxes are described by their maximum and minimum x and y coordinates.
\end{figure}

\begin{figure}
  \caption{Minimal Example of the COCO Annotation Format}
  \label{fig:coco}
  \par\noindent\rule{\textwidth}{0.5pt}
  \begin{\codefigsize}
  \lstset{literate={∆}{{$\Delta$}}1}
  \begin{lstlisting}
    {
      "annotations": [
        {
          "area": 600,
          "bbox": [
            100,
            50,
            20,
            30
          ],
          "category_id": 186,
          "id": 1,
          "image_id": 1,
          "iscrowd": 1
          "tags": {...}
        }
      ],
      "categories": [
        {
          "id": 186,
          "name": "∆",
          "supercategory": "Greek"
        }
      ],
      "images": [
        {
          "bln_id": 1,
          "date_captured": null,
          "file_name": "images/image.png",
          "img_url": "images/image.png",
          "width": 1000,
          "height": 2000,
          "id": 1,
          "license": 2
        },
      ],
      "licenses": [
        {
          "id": 2,
          "name": "CC-BY-NC 3.0.",
          "url": "https://creativecommons.org/licenses/by/3.0/"
        },
      ]
    }
  \end{lstlisting}
  \end{\codefigsize}
  \par\noindent\rule{\textwidth}{0.5pt}
  The minimum required information for a single image with a single instance of the $\Delta$ glyph in the COCO file format. The bounding boxes in this format are output in the COCO bounding box format \seefig{cocobbox}, in which bounding boxes are described by their minimum x and y coordinates and their width and height.
\end{figure}
