\begin{figure}
\caption{Minimal Example of the COCO Annotation Format}
\centering
\par\noindent\rule{\textwidth}{0.5pt}
\begin{\codefigsize}
\lstset{literate={∆}{{$\Delta$}}1}
\begin{lstlisting}
{
  "annotations": [
    {
      "area": 600,
      "bbox": [
        100,
        50,
        20,
        30
      ],
      "category_id": 186,
      "id": 1,
      "image_id": 1,
      "iscrowd": 1
      "tags": {...}
    },
  ],
  "categories": [
    {
      "id": 186,
      "name": "∆",
      "supercategory": "Greek"
    }
  ],
  "images": [
    {
      "bln_id": 1,
      "date_captured": null,
      "file_name": ...,
      "img_url": ...,
      "width": width (integer),
      "height": height (integer),
      "id": 1,
      "license": 2
    },
  ],
  "licenses": [
    {
      "id": 2,
      "name": "CC-BY-NC 3.0.",
      "url": "https://creativecommons.org/licenses/by/3.0/"
    },
  ]
}
\end{lstlisting}
\end{\codefigsize}
\par\noindent\rule{\textwidth}{0.5pt}
The minimum required information for a single image with a single instance of the $\Delta$ glyph in the COCO file format. Elipsis (\ldots) represent information removed for conciseness.
\label{fig:coco}
\end{figure}
