\documentclass[12pt,a4paper,final]{article}
\usepackage[a4paper, total={6in, 8.5in}]{geometry}
\setcounter{secnumdepth}{0}
\usepackage[utf8]{inputenc}
\usepackage{amsmath}
\usepackage{amsfonts}
\usepackage{amssymb}
\usepackage{sectsty}
\allsectionsfont{\centering}
\usepackage{textcomp}
\usepackage{pdfpages}
\usepackage{graphicx}
\usepackage{subcaption}
\graphicspath{{../figures/}}
\usepackage{hyperref}
\usepackage{enumitem}
\usepackage{multicol}
\usepackage{forest}
\usepackage{float}
\usepackage{mathtools}
\usepackage{todonotes}
\usepackage{fancyhdr}
\usepackage{tikz}
\usepackage[style=numeric-comp,sorting=nyt]{biblatex}
\makeatletter

\Hy@AtBeginDocument{
  \def\@pdfborder{0 0 1}
  \def\@pdfborderstyle{/S/U/W 1}
}
\makeatother
\hypersetup{
    colorlinks=true,
    citecolor=black,
    citebordercolor=black,
    linkcolor=blue,
    filecolor=magenta,
    urlcolor=cyan
}

\urlstyle{same}

\newcommand{\code}{\texttt}
\renewcommand{\O}[1]{$\mathcal{O}(#1)$}
\renewcommand{\o}[1]{$o(#1)$}
\newcommand{\T}[1]{$\Theta(#1)$}
\newcommand{\W}[1]{$\Omega(#1)$}
\newcommand{\w}[1]{$\omega(#1)$}
\newcommand{\command}[1]{\\\\\code{#1}\\\\\noindent}

\newcommand\qed{$\blacksquare$}

\DeclarePairedDelimiter\ceil{\lceil}{\rceil}
\DeclarePairedDelimiter\floor{\lfloor}{\rfloor}
\DeclarePairedDelimiter\abs{\lvert}{\rvert}

\addbibresource{../bibliography.bib}

\newcommand{\practicalInfoVar}{CS5199 - Context Survey}
\newcommand{\titleVar}{Thesis Skeleton, Context Survey, \& Work Plan}
\newcommand{\subTitleVar}{\\}
\newcommand{\dateVar}{October 14, 2022}
\newcommand{\matricNumberVar}{180008859}

\begin{document}
\title{\practicalInfoVar:\\\titleVar\subTitleVar}
\author{\matricNumberVar}
\date{\dateVar}
\maketitle

\pagestyle{fancy}
\fancyhf{}
\lhead{\practicalInfoVar}
\rhead{| \matricNumberVar}
\rfoot{Page \thepage}

\newpage
\tableofcontents
\newpage

\setlength{\parindent}{0pt}
\setlength{\parskip}{1em}

\section{Proposed Thesis Skeleton}
% table of contents with all the chapter and section headings. These will form the skeleton of your thesis and ensure that your report is properly structured;
\begin{itemize}[label={}]
  \item Abstract
  \item Acknowledgements
  \item Declaration
  \item Table of Contents
  \item Introduction
  \item Context Survey
  \item - Binarization
  \item - Glyph Grouping / Line Formation
  \item - Classification
  \item - Existing Methodologies
  \item Objectives
  \item - Primary
  \item - Secondary
  \item Design / Implementation
  \item - [Section for each method attempted]
  \item Evaluation
  \item - [Section for each method attempted]
  \item Conclusion
  \item References
  \item Appendix
\end{itemize}

\newpage
\section{Context Survey}
% a mostly complete review of related work (literature review). This is normally 5-10 pages long and will include citations to most important papers on this topic and explain how they relate to the task;
\subsection{Binarization}
As many of the most common, commercial, and competitive optical character recognition (OCR) models and methods require or work better with a binarized image \cite{Gupta, Smith, Bar-Yosef2005, Bar-Yosef2007} (also called a bi-level image), it is important to explore binarization as a method to improve the results of the pipeline.
Various methods are utilized for binarization of historical manuscripts, including thresholding \cite{Bar-Yosef2005, Bar-Yosef2007}, pixel clustering \cite{Bera}, and convolutional neural networks (CNNs) \cite{Dhali2019, Dhali2020, Xiong}. In each of these methods, the goal is the same, to separate the markings on the document from the rest of the image. This usually involves generating a second image where the ink is marked by a black pixel, with the rest of the image being white.

In the case of papyrus, this is not a trivial task, requiring techniques to have both high precision and recall to generate useful bi-level images to pass to further steps of the OCR pipeline. The strongest methods for binarization are able to not only differentiate ink from papyrus, but also extraneous text markings such as ruler marks and annotations added by historians as well as elements in the image that are either edited in after the image is taken or elements other than the papyrus.

Bera et al.\cite{Bera} propose an approach to binarization that involves using clustering algorithms such fuzzy C-means, K-medoids and K-means++ to form clusters of pixels into groups representing foreground and background elements. This method also makes use of normalization, to remove noise and shadow, and thresholding, maximizing inter cluster distance and minimizing inter cluster distance, thus generating very specific clusters. Pixels are labeled by the clusters and then the final output is determined by a decision algorithm that takes in the labels from the three clustering algorithms. This approach generates an F-measure result of $76.84$ and a peak signal to noise ratios of $15.31$, both measured against the 2018 DIBCO (Document Image Binarization COntest) dataset\cite{DIBCO2018}.

Dhali et al.\cite{Dhali2019} and Xiong et al.\cite{Xiong} both utilize CNNs for this task using manually annotated data in combination with transfer learning to retrain existing models to generate new networks with F-measure results of $86.7$ and $92.81$ respectively and peak signal to noise ratios of $21.3$ and $17.56$ respectively (all measured against DIBCO'18). While it has a lower F-measure on the whole of the DIBCO'18 dataset, the network proposed by Dhali et al., BiNet, was trained more specifically on degraded manuscripts and is shown to also be able to remove the extraneous elements from the image including rulers, machine-printed color-calibrators, and picture frames.

By combining and contrasting these methods against each other, a high quality, bi-level representation can be generated that can be passed to the next step of an OCR pipeline to bound and classify each glyph with more accuracy than would be expected with an unaltered image.

\subsection{Glyph Grouping / Bounding Boxes}
Precise bounding boxes are key to the success of every classification method, as removing noise from the edges of the glyphs to increase the signal to noise ratio of the signal being input into the classification algorithm. Similarly, to utilize a language model to help assist in classification, and to generate document level transcriptions, a method for converting a list of bounding boxes into a list of lines, each with at least bounding box is required.

A method proposed by Garz et al.\cite{Garz} utilizes points of interest to find the spaces between lines, which requires no binarization, and can be preformed on the source image without first generating bounding boxes, which allows this method to generate potential bounding boxes by detecting clusters of points of interest (generated using Difference of Gaussian \cite{Lowe}). Points of interest are then used to generate bounding boxes using Density-Based Spatial Clustering of Applications with Noise (DBSCAN) \cite{Ester}, which in the work proposed by Garz et al.\cite{Garz} requires a manual estimation of neighborhood size to create clusters. However, it could be possible to automatically estimate this distance using a more basic bounding box finding algorithm. Points of interest are also utilized to generate lines by splitting the image on the lines of lowest energy (farthest from the points of interest), weighting horizonal movement as lower energy than vertical movement.
This method, analyzed on hand written text from all 60 pages of the \textit{Saint Gall} database has a line accuracy of $97.97\%$ over 1431 lines of text, some of which contain background noise, which may allow this method to be translated to papyrus.

Williams (2014)\cite{Williams2014} also proposes a method for line detection based on finding gaps in the vertical coordinates of the glyphs. By finding the potential centers of each glyph and then sorting all the potential centers by their vertical coordinate, significant gaps in the list are used to separate lines. For each line, Williams then takes the line of best fit, using regression, to determine a potential line. A second pass then separate glyphs that are outside of the standard distance by a certain amount. This method appears to work quite well for glyphs where many potential centers are known, such as for crowdsourced data or binarized data with centroids. However, it fails significantly when a line has enough curvature that two lines occupy the same vertical coordinates over the horizontal span of the image.

In contrast to the algorithmic approaches above, Ezra et al.\cite{Ezra} propose a supervised learning approach for transcription alignment and bounding. Using a CNN-RNN trained to classify glyphs on lines to generate results with an accuracy of $90.3\%$. They also achieved a mean bounding box overlap of $81.0\%$ using the same model.

Another potential bounding method that could be explored is utilizing binarization to generate blobs for each glyph and bounding each blob, which combined with any of the above methods could be used to create tight bounding boxes for use in classification as the bounding boxes could either be kept binarized, or converted back to the full color image depending on the classification method being used.

Other potential sources for line finding are Tesseract\cite{SmithTesseract} and Ocular\cite{Berg-Kirkpatrick}. Tesseract is a popular OCR engine with the ability to preform line finding \cite{SmithLines} using blobs on images that can be skewed, meaning the image quality is not lost due to de-skewing the images.

\subsection{Classification}

Bar-Yosef (2005)\cite{Bar-Yosef2005} and Bar-Yosef et al. (2007)\cite{Bar-Yosef2005} both utilize a simple thresholding algorithm to generate a initial binarized image to use for further classification. Bar-Yosef et al. expands on the work done by Bar-Yosef by eroding characters to generate 'structuring elements' for each glyph, unique elements of glyphs that can be overlaid over the binarized text to identify characters of each glyph class. Using  this method, Bar-Yosef et al. were able to achieve an accuracy of $94.65\%$ on a test set of 1477 aleph characters over 22 Hebrew documents. By combining the more advanced binarization methods proposed by Xiong et al. and Dhali et al. it may be possible to generate equivalent or better results on the more degraded, ancient Greek documents targeted by this research.

\subsection{Croudsourced Transcriptions}
A large portion of the research in ancient Greek manuscript OCR has been done utilizing crowdsourced citizen science, allowing papyrologists and non-papyrologists alike to annotate ancient Greek manuscripts utilizing various pipelines that allow users detect and/or classify Greek letters.\cite{Williams2014, Williams2015, Tabin, Atanasiu} These methods, while not helpful in generating a fully automated pipeline, are valuable for generating data for both training and testing of fully automated pipelines.

Like Dhali et al.\cite{Dhali2019}, Tabin\cite{Tabin} utilizes a manual facsimile generation process that allows a user to manually annotate pixels in images, which they use to preform OCR on directly, but that could also be utilized in generating more data to train on as Dhali et al.\cite{Dhali2019} proposed.

Another usage for crowdsourced OCR is proposed by Williams (2014) \cite{Williams2014}, in which the consensus transcriptions are compared to known Greek texts to determine if the crowdsourced transcription may be a part of a known text. This allows for error correction and filling in the gaps, as well as the potential for creation of more accurate training data, as damaged or partially missing glyphs could still be classified correctly if they can be matched to known glyphs from other texts.

This method for transcription, while not able to be directly integrated into a fully automated pipeline for OCR is a a potential valuable source for training and/or testing data, and could be utilized to verify or improve the results of a future pipeline.

\newpage
\section{Work Plan}
\todo{}
% a work plan for the rest of your dissertation period (week-by-week) indicating the main tasks and objectives you will need to tackle and when you will be doing this. This is usually in the form of a table, a Gantt chart, or similar.

\printbibliography[heading=bibintoc, title={References}]

\end{document}
